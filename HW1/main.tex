\documentclass{article}
\usepackage{amsmath}
\usepackage{listings} % for typesetting code
\usepackage{amsfonts}
\begin{document}

\section*{EX.1 Number Conversions}
\section*{ Calculations were made on paper }
\subsection*{a. \( 1100101110_2 \) to decimal}
\[
1100101110_2 = 814_{10}
\]

\subsection*{b. \( 11001000011101_2 \) to hexadecimal}
\[
11001000011101_2 = 321D_{16}
\]

\subsection*{c. \( 11001000011101_2 \) to octal}
\[
11001000011101_2 = 31075_8
\]

\subsection*{d. \( 458_{10} \) to binary}
\[
458_{10} = 111001010_2
\]

\subsection*{e. \( 6197_{10} \) to octal}
\[
6197_{10} = 14605_8
\]

\subsection*{f. \( 15816_{10} \) to hexadecimal}
\[
15816_{10} = 3DC8_{16}
\]

\subsection*{g. \( 245_8 \) to binary}
\[
245_8 = 010100101_2
\]

\subsection*{h. \( 5026_8 \) to decimal}
\[
5026_8 = 2582_{10}
\]

\subsection*{i. \( 437_8 \) to hexadecimal}
\[
437_8 = 23F_{16}
\]

\subsection*{j. \( 9FEA_{16} \) to binary}
\[
9FEA_{16} = 1001111111101010_2
\]

\subsection*{k. \( 9FEA_{16} \) to octal}
\[
9FEA_{16} = 117752_8
\]

\subsection*{l. \( 1D4C_{16} \) to decimal}
\[
1D4C_{16} = 7500_{10}
\]

\subsection*{m. \( 3G8_{19} \) to 13-base notation}
\[
3G8_{19} = 836_{13}
\]


\section*{EX.2 Signed Binary to Decimal Conversion}

\subsection*{a. \( 0000\ 1000\ 0001\ 1010\ 0110\ 0101\ 0111\ 0011 \)}

Since the binary number starts with a 0, it is positive. We convert it directly to decimal:

\[
(0 \times 2^{31}) + (0 \times 2^{30}) + (0 \times 2^{29}) + (0 \times 2^{28}) +\] \[(1 \times 2^{27}) + (0 \times 2^{26}) + (0 \times 2^{25}) + (0 \times 2^{24}) + (0 \times 2^{23}) + (0 \times 2^{22}) + (0 \times 2^{21}) + (1 \times 2^{20}) +\]\[ (1 \times 2^{19}) + (0 \times 2^{18}) + (1 \times 2^{17}) + (0 \times 2^{16}) + (0 \times 2^{15}) + (1 \times 2^{14}) + (1 \times 2^{13}) + (0 \times 2^{12}) +\]\[ (0 \times 2^{11}) + (1 \times 2^{10}) + (0 \times 2^{9}) + (1 \times 2^{8}) + (0 \times 2^{7}) + (1 \times 2^{6}) + (1 \times 2^{5}) + (1 \times 2^{4}) +\] \[(0 \times 2^{3}) + (0 \times 2^{2}) + (1 \times 2^{1}) + (1 \times 2^{0}) = (135947635)_{10}\]



\subsection*{b. \( 1000\ 1000\ 0001\ 1010\ 0110\ 0101\ 0111\ 0011 \)}

Since the binary number starts with a 1, it is negative in two's complement format. To find the decimal value, we follow these steps:

\begin{enumerate}
    \item Find the two's complement (invert the bits and add 1).
    \item Convert the resulting binary number to decimal.
    \item Negate the result.
\end{enumerate}

The binary number:

\[
1000\ 1000\ 0001\ 1010\ 0110\ 0101\ 0111\ 0011_2
\]

Step 1: Invert the bits:
\[
0111\ 0111\ 1110\ 0101\ 1001\ 1010\ 1000\ 1100_2
\]

Add 1:
\[
0111\ 0111\ 1110\ 0101\ 1001\ 1010\ 1000\ 1100_2 + 1 = 0111\ 0111\ 1110\ 0101\ 1001\ 1010\ 1000\ 1101_2
\]

Step 2: Convert to decimal:
\[
0111\ 0111\ 1110\ 0101\ 1001\ 1010\ 1000\ 1101_2
\]

\[
(0 \times 2^{31}) + (1 \times 2^{30}) + (1 \times 2^{29}) + (1 \times 2^{28}) + (0 \times 2^{27}) + (1 \times 2^{26}) + (1 \times 2^{25}) + (1 \times 2^{24}) + (1 \times 2^{23}) + \]\[(1 \times 2^{22}) + (1 \times 2^{21}) + (0 \times 2^{20}) + (0 \times 2^{19}) + (1 \times 2^{18}) + (0 \times 2^{17}) + (1 \times 2^{16}) + (1 \times 2^{15}) +\]\[ (0 \times 2^{14}) + (0 \times 2^{13}) + (1 \times 2^{12}) + (1 \times 2^{11}) + (0 \times 2^{10}) + (1 \times 2^{9}) + (0 \times 2^{8}) + \]\[(1 \times 2^{7}) + (0 \times 2^{6}) + (0 \times 2^{5}) + (0 \times 2^{4}) + (1 \times 2^{3}) + (1 \times 2^{2}) + (0 \times 2^{1}) + (1 \times 2^{0}) = (2011536013)_{10}
\]



Step 3: Negate the result:

\[
-2011536013
\]

\subsection*{EX.3}

\subsection*{(a) \(12345 \& 5013\)}

\begin{align*}
12345_{10} &= 0011000000111001_2 \\
5013_{10} &= 0001001110010101_2 \\
12345 \& 5013 &= 0011000000111001_2 \& 0001001110010101_2 \\
&= 0001000000010001_2 \\
&= 4113_{10}
\end{align*}

\subsection*{(b) \(432 \, | \, -502\)}

\begin{align*}
432_{10} &= 0000000110110000_2 \\
502_{10} &= 0000000111110110_2 \\
-502_{10} &= 1111111000001010_2 \\
432 \, | \, -502 &= 0000000110110000_2 \, | \, 1111111000001010_2 \\
&= 1111111110111010_2 \\
&= 1111111110111010_2 \text{ (in two's complement, invert and add 1)} \\
&= 0000000001000101_2 + 1 \\
&= 0000000001000110_2 \\
&= -70_{10}
\end{align*}


\subsubsection*{(c) \( 19 \hat{} (\sim 67) \)}


\begin{align*}
19_{10} &= 0000000000010011_2 \\
67_{10} &= 0000000001000011_2 \\
\sim 67_{10} &= 1111111110111100_2 \\
19 \, ^ \, \sim 67 &= 0000000000010011_2 \,  ^ \, 1111111110111100_2 \\
&= 1111111110101111_2 \\
&= 1111111110101111_2 \text{ (in two's complement, invert and add 1)} \\
&= 0000000001010000_2 + 1 \\
&= 0000000001010001_2 \\
&= -81_{10}
\end{align*}

\subsection*{(d) \(-178 \, >> \, 2\)}

\begin{align*}
178_{10} &= 0000000010110010_2 \\
-178_{10} &= 1111111101001110_2 \\
-178 \, >> \, 2 &= 1111111101001110_2 \, >> \, 2 \\
&= 1111111111010011_2 \\
&= 1111111111010011_2 \text{ (in two's complement, invert and add 1)} \\
&= 0000000000101100_2 + 1 \\
&= 0000000000101101_2 \\
&= -45_{10}
\end{align*}

\subsection*{(e) \( ( \sim 178 + 1) \, >>> \, 4 \)}

\begin{align*}
178_{10} &= 0000000010110010_2 \\
\sim 178_{10} &= 1111111101001101_2 \\
(\sim178 + 1) &= 1111111101001101_2 + 1 \\
&= 1111111101001110_2 \\
(\sim 178 + 1) \, >>> \, 4 &= 1111111101001110_2 \, >>> \, 4 \\
&= 111111110100_2 \\
&= 268435444_{10}
\end{align*}

\section*{EX.4Arithmetic Operations in 8-bit Two's Complement Notation}

\subsection*{(a) \(88 - 50\)}

1. Convert 88 and 50 to their 8-bit binary representations:
\[
88_{10} = 01011000_2
\]
\[
50_{10} = 00110010_2
\]

2. Convert 50 to its two's complement (invert and add 1):
\[
50_{10} = 00110010_2 \quad \rightarrow \quad \text{invert} \rightarrow 11001101_2 \quad \rightarrow \quad \text{add 1} \rightarrow 11001110_2
\]

3. Perform the addition:
\[
01011000_2 + 11001110_2 = 100100110_2
\]

4. Consider only the least significant 8 bits:
\[
100100110_2 \rightarrow 00100110_2
\]

5. Convert the result back to decimal:
\[
00100110_2 = 38_{10}
\]

Thus, \(88 - 50 = 38\) is mathematically sound and valid.

\subsection*{(b) \(88 + 50\)}

1. Convert 88 and 50 to their 8-bit binary representations:
\[
88_{10} = 01011000_2
\]
\[
50_{10} = 00110010_2
\]

2. Perform the addition:
\[
01011000_2 + 00110010_2 = 10001010_2
\]

3. Consider only the least significant 8 bits:
\[
10001010_2
\]

4. Convert the result back to decimal (two's complement):
\[
10001010_2 \quad \text{(negate the bits)} \rightarrow 01110101_2 \quad \rightarrow \quad \text{add 1} \rightarrow 01110110_2 = 118_{10} \rightarrow -118_{10}
\]

Thus, \(88 + 50 = -118\) in 8-bit two's complement notation. The result is not mathematically valid in standard arithmetic due to overflow.
\subsection*{EX.5 Java Hello World }

\lstset{language=Java} % set the language for the code
\begin{lstlisting}
public class HelloWorld {
    public static void main(String[] args) {
    String UserName = "Vahagn";
       String UserId = "MU6613477";
        
       System.out.println("Hello user. My name is " 
                       + UserName + ". My ID is " + UserId);
    }
}
\end{lstlisting}
\end{document}

